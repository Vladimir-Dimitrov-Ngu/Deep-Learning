\documentclass[a4paper,12pt]{article}

%%% Работа с русским языком
\usepackage{cmap}					% поиск в PDF
\usepackage{mathtext} 				% русские буквы в фомулах
\usepackage[T2A]{fontenc}			% кодировка
\usepackage[utf8]{inputenc}			% кодировка исходного текста
\usepackage[english,russian]{babel}	% локализация и переносы
\renewcommand{\baselinestretch}{1.5} % интервал между строками
%%% Дополнительная работа с математикой
\usepackage{amsfonts,amssymb,amsthm,mathtools} % AMS
\usepackage{amsmath}
\usepackage{icomma} % "Умная" запятая: $0,2$ --- число, $0, 2$ --- перечисление

%% Номера формул
%\mathtoolsset{showonlyrefs=true} % Показывать номера только у тех формул, на которые есть \eqref{} в тексте.

%% Шрифты
\usepackage{euscript}	 % Шрифт Евклид
\usepackage{mathrsfs} % Красивый матшрифт

%% Свои команды
\DeclareMathOperator{\sgn}{\mathop{sgn}}

%% Перенос знаков в формулах (по Львовскому)
\newcommand*{\hm}[1]{#1\nobreak\discretionary{}
	{\hbox{$\mathsurround=0pt #1$}}{}}

%%% Работа с картинками
\usepackage{graphicx}  % Для вставки рисунков
\graphicspath{{images/}{images2/}}  % папки с картинками
\setlength\fboxsep{3pt} % Отступ рамки \fbox{} от рисунка
\setlength\fboxrule{1pt} % Толщина линий рамки \fbox{}
\usepackage{wrapfig} % Обтекание рисунков и таблиц текстом

%%% Работа с таблицами
\usepackage{array,tabularx,tabulary,booktabs} % Дополнительная работа с таблицами
\usepackage{longtable}  % Длинные таблицы
\usepackage{multirow} % Слияние строк в таблице


%%% Заголовок
\author{Vladimir Dimitrov}
\title{Lecture DL}
\date{}


\begin{document}
	\maketitle
	\newpage
	\tableofcontents
	\newpage
	\section{Lecture 1}
	\textbf{Полносвязные слои (fully conected, FC)}: представляет набор линейных моделей
	\begin{itemize}
		\item На входе n чисел, на выходе m
		\item $x_1, \ldots, x_n$  - входы
		\item $z_1, \ldots, z_m$ - выходы
		\item Каждый выход - линейная модель над входами:
		$$
			z_{j}=\sum_{i=1}^{n} w_{j i} x_{i}+b_{j}
		$$
	\end{itemize} 
	
	В такое модели nm параметров - надо много данных. После FC применяется нелинейная модель.
	
	
	
	\textbf{Нелинейный преобразования}:
	\begin{itemize}
		\item Сигмоида: $f(x)=\frac{1}{1+\exp (-x)}$ (проблема затухания градиента)
		\item ReLU (Rectified Linear Unit): $f(x)=\max (0, x)$ 
		
	\end{itemize}
	
	Нелинейность можно брать одну и тоже.
	
	\textbf{Теорема Цыбенко}
	
	Пусть существует g(x) - непрерывная функция, тогда можно построить двухслойную нейронную сеть, приближающую g(x) с любой заранее заданной точностью. 
	
	\newpage
	\section{Lecture 2}
	\textbf{Обучение нейронной сети}
	
	Критерий остановы: ошибка на тестовой выборки перестает убывать (немного отличается чем в ML).
	
	В основе лежит метод обратной распространения ошибки (Backpropagation). Просто перемножение матриц.
	
	\textbf{Полносвязные сети для изображений}
	
	\textbf{Данные MNIST}
	\begin{itemize}
		\item Изображение 28 * 28
		\item Изображения центрированы 
		\item 60000 объектов в обучающей выборке
	\end{itemize}
	
	Полносвязные сети плохо работают с картинками, так как: 
	\begin{itemize}
		\item Много параметров
		\item Легко переобучиться 
		\item Не учитывает специфику изображений
		\item Один из способов бороться с переобучением - сократить число параметров
	\end{itemize}

	\textbf{Свертки}
	
	\begin{itemize}
		\item Операция свертки выявляет наличие паттерна, который задается фильтром
		\item Чем сильнее на участке изображения представлен паттерн, тем больше будет значение свертки
	\end{itemize}
	
	Фильтр Собеля (Детекция границ):
	$$\left[\begin{array}{rrr}+1 & 0 & -1 \\ +2 & 0 & -2 \\ +1 & 0 & -1\end{array}\right]$$
	
	
	
\end{document}